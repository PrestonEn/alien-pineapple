\thispagestyle{empty}
\section{Abstract}

\begin{doublespace}
Community detection in complex networks has been one of the most popular topics in several fields. Communities, also referred to as clusters, are typically defined as subgraphs with a higher probability of sharing edges with nodes within that subgraph that the rest of the graph. Several algorithms for identifying these structures have been developed. Genetic algorithms have long been used to tackle large scale combinatorial optimization problems, and have appeared sporadically throughout the literature for them community detection problem. For this work, four genetic approaches have been selected, using a variety of representations, fitness measures and clustering approaches. After implementing the designs as they are described, they were tested against a series of benchmarks. These include the Girvan-newmann and LFR synthetic benchmarks, random graphs, and several well studied real world networks. The obtained clusterings are compared using several measures of clustering accuracy, Variation of information, Adjusted Rand index, and normalized mutual information. The performance of the selected approaches show that GA's can be fielded as an effective approach in estimating community structures on real data.
\end{doublespace}

