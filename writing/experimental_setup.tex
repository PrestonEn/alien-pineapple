\chapter{Methodology}
Here we describe the selected algorithms and their parameters in detail. We also discuss the nature of the benchmarks and real world data, giving a summary of the range of tests to be performed. 

\section{Selected Algorithms}
\subsection{Tasgin-Bingol}
One of the earliest implementations of a genetic algorithm for the network clustering problem \cite{Tasgin2006}, Tasgin and Bingol's approach is an example of one of the more naive approaches. Each individual is represented as an array of size $n$, each index corresponding to a node of the input graph. The population is initialized with a random walk over the graph. . It utilizes modularity as its fitness function. Its mutation operator is simple. For the selected node $i$, select a member of $\Gamma(i)$, and set $i$ to be in its community.

The table \ref{table:1} is an example of referenced \LaTeX elements.

\begin{table}[h!]
	\centering
	\begin{tabular}{| c | c | c | c | c |}
		\hline
		Population Size & Generations & Crossover Rate & Mutation Rate  & Cleaning Rate \\ [0.5ex] 
		\hline
		1 & 6 & 87837 & 787 & 6  \\ 

		\hline
	\end{tabular}
	\caption{Tasgin-Bingol's default parameters}
	\label{table:1}
\end{table}

The



\subsection{GA-Net}
\cite{Pizzuti2008}
\begin{table}[h!]
	\centering
	\begin{tabular}{| c | c | c | c | c |}
		\hline
		Population Size & Generations & Crossover Rate & Mutation Rate  & Cleaning Rate \\ [0.5ex] 
		\hline
		1 & 6 & 87837 & 787 & 6  \\ 
		
		\hline
	\end{tabular}
	\caption{GA-Net's default parameters}
	\label{table:1}
\end{table}


\subsection{GACD}
\cite{Shi2009}
\begin{table}[h!]
	\centering
	\begin{tabular}{| c | c | c | c | c |}
		\hline
		Population Size & Generations & Crossover Rate & Mutation Rate  & Cleaning Rate \\ [0.5ex] 
		\hline
		1 & 6 & 87837 & 787 & 6  \\ 
		
		\hline
	\end{tabular}
	\caption{GACD's default parameters}
	\label{table:1}
\end{table}


\subsection{GALS}
\cite{liu2013genetic}
\begin{table}[h!]
	\centering
	\begin{tabular}{| c | c | c |}
		\hline
		Population Size & Generations  & a \\ [0.5ex] 
		\hline
		1 & 6 & a \\ 
		
		\hline
	\end{tabular}
	\caption{GALS' default parameters}
	\label{table:1}
\end{table}


\section{The Girvan-Newman Benchmark}



\section{The LFR Benchmark}
\begin{tabular}{|c|}
	\hline 
	Number of Nodes \\ 
	\hline 
	Maximum Degree \\ 
	\hline 
\end{tabular} 

\cite{Lancichinetti2008}


\section{Benchmark Generator}
All synthetic networks were created using Andrea Lancichinetti's benchmark generation applications 

\section{Real World Data}
\subsection{Zachary's Karate Club}
Zachary's Karate Club \cite{Zachary1977} is one of the most widely used networks used to show an algorithm can effectively identify a set of communities. The graph shows the reported relationships between members of a martial arts club, after a schism was formed by a conflict between the lead instructor and the owner of the club. Though small, the graph shows the expected power law of degree distribution for 

\begin{figure}[!htb]
	\begin{center}
		\includegraphics[scale=.75]{images/kachary.png}
	\end{center}
	\caption{The flow of a typical genetic algorithm.}
	\label{logo}
\end{figure}

\subsection{Dolphins}
\cite{Lusseau2003}


\subsection{College Football}
\cite{Girvan2002}


\subsection{Political Blogs}
\cite{Adamic2005}


\section{Experiment Description}


\subsection{Variation of Information}
\cite{Marina2007}

\subsection{Normalized Mutual Information}
The Normalized Mutual Information (NMI) of a community partition 

\subsection{Rand and Adjusted Rand Index}
\cite{rand1971}