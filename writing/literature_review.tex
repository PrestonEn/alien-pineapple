\chapter{Literature Review}




\section{Methods of Community Detection}
While the number of approaches for community detection and graph partitioning is vast, most applications employ one of a few popular methods. In this section we introduce other techniques beyond genetic algorithms, as well as other evolutionary methods not selected for this work. 

\subsection{Hierarchical Clustering}
Hierarchical methods are the traditional method of community detection. Early implementations gained popularity by not requiring any previous knowledge of the network itself, a condition of algorithms of the related \textit{graph partitioning} problem. In general, hierarchical clustering, applied to graph or more traditional vector data, allows the discovery of a multilevel sequence of clusters. In the case of graphs, this presents as small sub-communities composing a larger community.


\subsection{Girvan and Newmann's Divisive Method}
\cite{Girvan2002}




\subsection{Spectral Clustering}

%spectral methods

%divisive

%cite modularity max/sim anneal
\subsection{Maximization Methods}


%raddachi

%methods maximizing centrality values


\subsection{Genetic Algortihms}





\cite{Brandes}




%genetic approaches

%surveys comparing algorithms
\section{Evaluating Community Detection Algorithms}




