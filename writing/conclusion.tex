\chapter{Conclusion}
\label{ch:conclusion}

The continued interest in the study of community structures of complex networks has resulted in a multitude of approaches to the problem. Genetic algorithms have proved themselves to be effective in dealing with optimization problems by levering biologically inspired operators to explore a large search space. While GAs are a fairly under represented class of algorithms in the community detection literature, that by no means they are not worth further investigation.

In this work, four GAs for the community detection problem were selected and implemented as described in their respective publications. Using several parameter sets for each, as proposed in each paper and as selected for this work, their results were compared over a battery of input data.
We presented two classes of well understood benchmarks, GN, and LFR. By knowing the community structure of each example graph, a direct comparison of performance is able to be made between each approach. For both cases, we show that each algorithm can detect the structure of networks with a reasonable degree of accuracy up to a point, with those algorithms using local search and guided initialization methods outperforming the others by a wide margin. In particular, the use of random mutation is often destructive. The locus-adjacency representation allows for a much greater variety of crossover strategies by ensuring community partitions are comparable. While experiments were limited to unweighted graphs, these approaches are extensible to weighted data, or even edges with metadata associated with nodes and edges by simply extending the fitness functions. Other networks may be dynamic, or contain overlapping community structures. Little exploration has been done in applying GA to these particular classes of the community detection problem.

The ubiquitousness of modularity as a fitness function has received much attention in literature, and results on the upper values of the LFR-size experiments show the sharp decline in fitness, even for the most effective method, GALS. More work is needed to understand when fitness functions are appropriate for a particular network.

The methods tested, particularly Tasgin and GA-Net, rely on a fairly large set of parameters. There are also several other models of GA not explored, such as island models. These should be selected carefully for each specific network, and presenting results on a wider scope of networks can be considered for future evaluation. A close examination of running time was not carried out, however it is clear that the use of local search operators come with a large penalty to running time, but with significant performance gains. More work should be carried out to analyze and improve the marginal gene approach used by GALS, and explore other approaches.