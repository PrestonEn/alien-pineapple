\chapter{Introduction}
\label{ch:intro:introduction}
Networks have long been used as a method for representing relations in data. Graph theory has its origins in Euler's 1736 paper proving a solution to the K\"{o}nigsberg's bridges problem. As research as been done into the properties of graphs, they have proven to be an excellent way of representing a wide variety of systems in many fields. Social, biological, communications and information can be represented as graphs, with many other fields of study making use of them as well. With the use of these representations in modern applications, their size and complexity has grown immensely. Where in the past a sociologist may have curated a dataset of social connections by hand, data can now be scrapped from global social networking platforms at scales of millions to billions of nodes. 

These large scale graphs showing the interactions of real systems can not be looked at the same way as structured graphs, such as lattices or regular graphs. Real networks are intensely chaotic. They are closer to random graphs, formalized as the Erd\H{o}s Renyi model. The probability of two nodes being connected is equal for all pairs of nodes. 

\section{The Problem of Community Detection in \\Complex Networks}

\section{Thesis Structure}