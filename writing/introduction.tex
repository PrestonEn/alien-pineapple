\chapter{Introduction}
\label{ch:intro:introduction}
Networks have long been used as a method for representing relations in data. Graph theory has its origins in Euler's 1736 paper proving a solution to the K\"{o}nigsberg's bridges problem. As research has been done into the properties of graphs, they have proven to be an excellent way of representing a wide variety of systems in many fields. Social, biological, communications and information can be represented as graphs, with many other fields of study making use of them as well. With the use of these representations in modern applications, their size and complexity has grown immensely. Where in the past a sociologist may have curated a dataset of social connections by hand, data can now be scraped from global social networking platforms at scales of millions to billions of nodes. 

One of the main structures of interest in such networks are communities, also referred to as modules or clusters. These are densely connected groups of nodes, which indicate some inherent order in the network. They can show groups of friends or coworkers in social networks, closely related protein-protein interactions, authors of papers working in similar fields and other explainable patterns in seemingly chaotic networks. Several approaches have been implemented to tackling the task of \textit{community detection}.\textit{Genetic algorithms} have shown to be fairly efficacious at discrete optimization tasks, and have been applied to graph clustering and related problems. In this study, we implement four genetic algorithms from literature and compare their performance against real world data and synthetic benchmarks to evaluate their effectiveness.


\pagebreak

\section{The Problem of Community Detection in \\Complex Networks}
Community detection in networks is analogous to clustering in vector data \cite{Peel2016}. Given a network, the goal is to identify groups of nodes that represent tightly connected structures in a larger graph. These tightly connected components are common in real networks and are vital in the study of these systems, particularly at scale.

\section{Thesis Structure}

The organization of this thesis is as follows. Chapter 2 introduces the foundational terms and concepts in graph theory and complex networks and a summary of approaches to the problem of community detection and related tasks. This is followed by a brief introduction to genetic algorithms. Chapter 3 contains a more in depth analysis of previous research done on the problem of community detection, with sections introducing each of the algorithms selected for study. The methodology used for comparing the selected algorithms, including the details regarding libraries, data sources and benchmarks is covered in chapter 4. Two types of data were used to test the algorithms, which can be classified as synthetic and real world data. These two categories are each reported in chapters 5 and 6. Chapter 7 contains discussion regarding obtained results and exploration of ideas for future work.